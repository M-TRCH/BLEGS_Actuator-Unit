\documentclass[a4paper,12pt]{article}

% การตั้งค่าภาษาไทยและการตัดคำ
\usepackage{fontspec}
\usepackage{xunicode}
\usepackage{xltxtra}
\usepackage{geometry}
\usepackage{fancyhdr}
\usepackage{amssymb} % สำหรับช่อง Checkbox
\usepackage{enumitem}
\usepackage{titlesec}

% ตั้งค่าฟอนต์ภาษาไทย (ต้องมีฟอนต์ TH Sarabun New ในเครื่อง หรืออัพโหลดขึ้น Overleaf)
\defaultfontfeatures{Mapping=tex-text}
\setmainfont{TH Sarabun New} % หรือเปลี่ยนเป็น Angsana New ตามที่มี
\XeTeXlinebreaklocale "th"
\XeTeXlinebreakskip = 0pt plus 1pt

% ตั้งค่าหน้ากระดาษ
\geometry{
    a4paper,
    top=20mm,
    bottom=20mm,
    left=20mm,
    right=20mm
}

% ส่วนหัวและท้ายกระดาษ
\pagestyle{fancy}
\fancyhf{}
\lhead{\textbf{โครงการ LGS (Light Guiding Shelf)}}
\rhead{แบบตรวจสอบการติดตั้ง (Checklist)}
\cfoot{หน้า \thepage}
\renewcommand{\headrulewidth}{0.4pt}

% สร้าง Checkbox
\newlist{todolist}{itemize}{2}
\setlist[todolist]{label=$\square$}

\title{\vspace{-2cm}\textbf{แบบตรวจสอบการติดตั้งและทดสอบระบบ}\\Light Guiding Shelf (LGS)}
\author{}
\date{\today}

\begin{document}

\maketitle
\vspace{-1.5cm}

\noindent
\textbf{ชื่อตู้/โซน:} \underline{\hspace{6cm}} \hfill \textbf{ผู้ติดตั้ง:} \underline{\hspace{5cm}} \\[0.5em]
\textbf{วันที่:} \underline{\hspace{6cm}} \hfill \textbf{ผู้ตรวจสอบ:} \underline{\hspace{5cm}}

\section{1. การเดินสายและโครงสร้างระบบ (Wiring \& Topology)}
\textit{ตรวจสอบการเชื่อมต่อทางกายภาพและค่าแรงดันไฟ}
\begin{todolist}
    \item \textbf{เส้นทางสาย LAN:} สายจาก Server $\rightarrow$ เสียบเข้า \textbf{PLC} $\rightarrow$ ขยายจาก PLC ไปเข้า \textbf{Modbus Gateway}
    \item \textbf{สีสายสัญญาณ RS485:}
    \begin{itemize}
        \item[$\circ$] สาย A: \textbf{สีน้ำเงิน (Blue)}
        \item[$\circ$] สาย B: \textbf{สีเขียว (Green)}
    \end{itemize}
    \item \textbf{ตรวจสอบแรงดันไฟ (Voltage):} วัดที่ \textbf{ปลายสาย (End of Line)} ของโมดูลตัวสุดท้าย
    \begin{itemize}
        \item[$\circ$] แรงดันต้องได้ $\mathbf{\ge 9.8}$ \textbf{V}
    \end{itemize}
\end{todolist}

\section{2. การตั้งค่า Modbus Gateway}
\textit{ตั้งค่าพารามิเตอร์ให้ถูกต้องเพื่อป้องกันข้อมูลชนกัน}
\begin{todolist}
    \item \textbf{Network Parameters:}
    \begin{itemize}
        \item[$\circ$] Port: \textbf{502}
        \item[$\circ$] Protocol: \textbf{Modbus TCP Protocol}
    \end{itemize}
    \item \textbf{Serial Parameters:}
    \begin{itemize}
        \item[$\circ$] Baud Rate: \textbf{9600} \hfill Data Bits: \textbf{8}
        \item[$\circ$] Parity: \textbf{None} \hfill Stop Bit: \textbf{1}
        \item[$\circ$] Flow Control: \textbf{None}
    \end{itemize}
    \item \textbf{Operating Mode (Advanced):}
    \begin{itemize}
        \item[$\circ$] Transfer Mode: \textbf{Auto Query Storage}
        \item[$\circ$] \textbf{Enable} RS485 Multi-host
        \begin{itemize}
             \item Wait time: \textbf{224 ms}
        \end{itemize}
        \item[$\circ$] \textbf{Enable} RS485 Bus Conflict Detection
        \begin{itemize}
             \item Send data only when RS485 bus is idle for: \textbf{20 ms}
        \end{itemize}
    \end{itemize}
\end{todolist}

\section{3. การตั้งค่า PLC (S7-1200 G2)}
\begin{todolist}
    \item \textbf{เลือกขนาดตู้ (Cabinet Size):} (ติ๊กเลือก Checkbox ในโปรแกรมให้ถูกต้อง)
    \begin{itemize}
        \item[$\circ$] $\square$ 60 ช่อง \quad $\square$ 68 ช่อง \quad $\square$ 40 ช่อง \quad $\square$ 80 ช่อง
    \end{itemize}
    \item \textbf{การซิงค์เวลา (Time Sync):}
    \begin{itemize}
        \item[$\circ$] เปิดใช้งาน NTP Client และซิงค์เวลากับ \textbf{คอมพิวเตอร์ที่ใช้ทดสอบ} ได้สำเร็จ
    \end{itemize}
\end{todolist}

\section{4. การทดสอบปุ่มกดหน้าตู้ (Functional Test)}
\textit{ทดสอบการกดปุ่มจริงที่ Control Panel}
\begin{todolist}
    \item \textbf{ปุ่มสีแดง (Red):} กดแล้ว \textbf{ไฟทุกสีติดครบทุกช่อง} (All ON)
    \item \textbf{ปุ่มสีเขียว (Green):} กดแล้ว \textbf{ไฟดับทุกช่อง} (All OFF)
    \item \textbf{ปุ่มสีน้ำเงิน (Blue):} กดแล้ว \textbf{ไฟสีแดงติดเรียงลำดับ} พร้อม \textbf{ปลดล็อกกลอน} (Sequence \& Unlock)
\end{todolist}

\section{5. การทดสอบผ่าน Software Tool}
\textit{ทดสอบสั่งงานผ่านโปรแกรม Test Tool (จำลองคำสั่งจาก Server)}
\begin{todolist}
    \item \textbf{สั่งคำสั่ง "All ON":} ผลลัพธ์ต้องเหมือนการกดปุ่มสีแดง
    \item \textbf{สั่งคำสั่ง "All OFF":} ผลลัพธ์ต้องเหมือนการกดปุ่มสีเขียว
    \item \textbf{สั่งคำสั่ง "Sequence/Unlock":} ผลลัพธ์ต้องเหมือนการกดปุ่มสีน้ำเงิน
\end{todolist}

\section{6. การตรวจสอบ Version Control (GitHub)}
\textit{ยืนยันว่า Source Code ทั้งหมดถูกจัดเก็บเข้าระบบแล้ว}
\begin{todolist}
    \item \textbf{Firmware ของ Module:} Code ที่เบิร์นลงบอร์ดถูก Commit ขึ้น GitHub แล้ว\\
    \textit{ระบุ Commit Hash/Tag:} \underline{\hspace{8cm}}
    \item \textbf{Program ของ PLC:} โปรเจค TIA Portal ที่โหลดลงเครื่องถูก Commit ขึ้น GitHub แล้ว\\
    \textit{ระบุ Commit Hash/Tag:} \underline{\hspace{8cm}}
\end{todolist}

\vspace{1cm}
\hrule
\vspace{0.5cm}

\textbf{บันทึกเพิ่มเติม / ปัญหาที่พบ:} \\[2.5cm]
\rule{\textwidth}{0.4pt}

\noindent
\textbf{ลงชื่อผู้ตรวจสอบ:} \underline{\hspace{6cm}} \hfill \textbf{วันที่:} \underline{\hspace{4cm}}

\end{document}
